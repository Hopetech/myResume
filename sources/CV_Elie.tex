%!TEX TS-program = xelatex
\documentclass[]{Elie-cv}

\begin{document}
\header{Elie}{TOURNIER}
       {\textbf{\color{white}OpenSource Software Engineer}}

\begin{aside}
  \section{About}
    608 chemin du bocage
    38980 VIRIVILLE
    FRANCE
    ~
    +33 6 37 41 44 96
    ~
    \href{mailto:tournier.elie@gmail.com}{\textbf{tournier.elie@gmail.com}}
    \href{https://fr.linkedin.com/in/elietournier}{linkedin.com/in/elietournier}
    \href{https://github.com/Hopetech}{github.com/hopetech}
     \href{https://gitlab.freedesktop.org/hopetech}{gitlab.freedesktop.org/hopetech}
  \section{Programming}
    C
    C++
    TCL
    MATLAB
    \LaTeX{} 
  \section{Revision control software}
    Git
  \section{Operating system}
    Linux
    Windows
  \section{Languages}
    French : Mother tongue
    English : \href{http://www.ets.org/toeic}{TOEIC} 895/990
    Italian : Notions   
\end{aside}

\section{Interests}

\textbf{Open-source}, \textbf{Graphics}, Compile, Information Communication Technologies, Healthcare, Image processing, Testing.

\section{Experience}

\begin{entrylist}
  \entry
    {01.2017-Now}
    {\href{https://www.collabora.com/}{Collabora}, Cambridge, UK}
    {Graphics software engineer}
    {\emph{ - Creating prototype of a trace testing \textbf{continuous integration} system for Mesa. Verifying the correctness of the image produce by the OpenGL driver. \\
    			- Adding a new backend in Virgl3d, a virtual 3D GPU use inside QEMU virtual machines. Implementing \textbf{lowering passes}. Transcompiling Mesa internal IR to GLSL or ESSL.\\
				- Adding double precision floating point support on GPU without hardware FPUs. Modifying the \textbf{Mesa compiler} to add new built-in functions.\\
				- Reviewing code from different components of Mesa.\\
				- Taking part of discussions the \textbf{Khronos OpenCL Tooling workgroup} to improve and upstream a SPIRV-LLVM IR translator to LLVM.}}

  \entry
    {05-09.2016}
    {\href{https://summerofcode.withgoogle.com/}{Google Summer of Code}, Remote}
    {Student software engineer}
    {\emph{Creating a library to emulate \textbf{IEEE754} double precision floating point.}}

  \entry
    {2013-2016}
    {\href{https://www.thalesgroup.com/en/microwave-imaging-sub-systems/radiology}{Thales Electron Devices}, Moirans, FRANCE}
    {Student software engineer}
    {\emph{ - Programming of images quality characterization tools in \textbf{C++}. \\
				- Designing and programming, in \textbf{TCL}, automated tests for X-ray detectors.}}

   \entry
    {04-08.2013}
    {\href{http://lpsc.in2p3.fr/index.php/en/}{LPSC}, Grenoble, FRANCE}
    {Technician training}
    {\emph{Designing and programming, in LabVIEW, of an optical polarimeter for the {\href{http://nedm.web.psi.ch/}{nEDM experiment}}.}}

\end{entrylist}

\section{Higher Education}

\begin{entrylist}
  \entry
    {2013–2016}
    {\href{http://www.telecom-physique.fr}{Engineering school}}
    {Telecom Physiques, Strasbourg, FRANCE}
    {Sandwich courses in Health ICT}
  \entry
    {2011–2013}
    {DUT Mesures Physiques}
    {Joseph Fourier University, Grenoble, FRANCE}
    {2-year course in Applied Physics and Measurement Processes – equivalent to a Higher National Diploma,\\
    			specialized in Instrumentation Techniques}
\end{entrylist}

\section{Presentation}

\begin{entrylist}
  \entry
    {Feb. 2019}
    {FOSDEM}
    {Brussels, Belgium}
    {What's new in the GPU virtual world?\\
    Overview of the project and plans for the future.}
   \entry
    {Oct. 2018}
    {X.Org Developer's Conference}
    {A Coruña, Spain}
    {What's new in the GPU virtual world?}
   \entry
    {Sept. 2016}
    {X.Org Developer's Conference}
    {Helsinki, Finland}
    {Implementation of a double floating point library in GLSL 1.30.\\
    Presenting the status of the FP64 emulation library.}
\end{entrylist}

\end{document}
